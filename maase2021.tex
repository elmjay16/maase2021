\documentclass[11pt]{beamer}
\usepackage[utf8]{inputenc}
\usepackage[T1]{fontenc}
\usepackage{amsmath}
\usepackage{amsfonts}
\usepackage{amssymb}
\usepackage{graphicx}
\usepackage{subfigure}
\usepackage{ulem}
\usetheme{CambridgeUS}
\begin{document}

\author{Lisa Jones}
\title{Mathematics in Cyber Security}
%\subtitle{}
%\logo{}
%\institute{}
%\date{}
%\subject{}
%\setbeamercovered{transparent}
%\setbeamertemplate{navigation symbols}{}
\frame[plain]{\maketitle}


\begin{frame}
\frametitle{What is discrete mathematics?}
  \begin{itemize}
  \item{Study of mathematical structures (sets + features) that do not vary ``smoothly'', but have distinct, separated values}
    \medskip
  \item{Examples: integers, graphs, statements in logic}
  \end{itemize}
\end{frame}



  \begin{frame}
\frametitle{What is a graph?}
\begin{definition} A \textbf{graph} is a pair  G = (V, E) ,
  where V is a set whose elements are called \textit{vertices} and
  E is a set whose elements are paired vertices, called \textit{edges}.
  If the elements of E are ordered pairs, we call the graph \textit{directed}; otherwise, we call it \textit{undirected}.
\end{definition}
\bigskip
\begin{definition}
  A \textbf{path} in a graph is a sequence of edges which joins a sequence of (distinct) vertices. Paths in directed graphs have an added restriction: the edges must all be directed in the same direction.
\end{definition}
  \end{frame}

  \begin{frame}
\frametitle{What is a graph?}
Insert picture of a directed graph with a path highlighted
\end{frame}

  
  \begin{frame}
\frametitle{Graphs in computer network defense}
Example: network intrusion detection
\bigskip
\begin{itemize}
  \item{Nodes - computers on a network (aka ``hosts''), labeled by IP addresses}
    \medskip
  \item{Edges - communication between hosts}
  \end{itemize}
  \end{frame}

   \begin{frame}
\frametitle{Graphs in computer network defense}
Example: Internet topology analysis
\bigskip
\begin{itemize}
  \item{Nodes - routers that advertise eBGP routes}
    \medskip
  \item{Edges - advertised connectivity between ``peered'' routers}
  \end{itemize}
  \end{frame}


  \begin{frame}
\frametitle{What is program analysis?}
  \begin{itemize}
  \item{ Automatically analyzing the behavior of computer programs regarding a property such as correctness, robustness, safety and liveness}
    \medskip
  \item{Can be \textit{static} (performed without running the program), \textit{dynamic}, or a hybrid of both}
      \end{itemize}
  \end{frame}

  \begin{frame}
\frametitle{Graphs in program analysis}
Control-flow graph:
\begin{itemize}
  \item{Nodes - \textit{basic blocks} of a program}
    \medskip
  \item{Edges - \textit{control flow} between basic blocks}
      \end{itemize}
  \end{frame}

  \begin{frame}
    Insert code of a program in a high level language
  \end{frame}

  \begin{frame}
    Insert code of program in assembly
  \end{frame}

  \begin{frame}
    Insert assembly program broken into basic blocks
  \end{frame}

  \begin{frame}
    Insert assembly program basic blocks redrawn as graph
  \end{frame}

  \begin{frame}
    \frametitle{Example program analyses}
    \begin{itemize}
    \item{\textbf{Control-flow analysis}: what functions get called}
      \medskip
    \item{\textbf{Data-flow analysis}: what ranges of values variables take} 
    \end{itemize}
  \end{frame}

  \begin{frame}
    Insert assembly program basic blocks redrawn as graph, add colors to nodes for functions
  \end{frame}

  
\begin{frame}
\frametitle{Example: finding an input to reach a desired state}
  Simple pseudo-code with if-else, while, abort() statement
  \end{frame}

\begin{frame}
\frametitle{Why is this hard?}
\begin{itemize}
\item{Undecidability}
  \medskip
  \item{Leaky abstractions}
    \medskip
  \item{Imprecise models}
    \end{itemize}
  \end{frame}
  
  
\begin{frame}
  \frametitle{Interlude: mathematical logic}
  \begin{itemize}
  \item{Formal language: set of symbols (aka ``alphabet'') + rules for putting them together into sentences (aka ``well-formed formulae'' (wff))}
    \medskip
  \item{Theory: a set of sentences in a formal language}
    \medskip
  \item{Decision problem: a yes-or-no question over a set of input values}
    % classified by the computational resources required to solve
\end{itemize}
\end{frame}

\begin{frame}
  \frametitle{Interlude: mathematical logic}
  \begin{itemize}
  \item{Propositional calculus: propositional variables + logical connectives \\
    + inference rules + axioms}
    \medskip
  \item{Boolean satisfiability (SAT): is there an assignment of variables to T/F such that the formula evaluates to T?}
    % NP-complete
  \end{itemize}
\end{frame}

\begin{frame}
  \frametitle{Example: propositional calculus}
  Insert truth table, highlight a satisfying assignment
\end{frame}

\begin{frame}
  \frametitle{Interlude: mathematical logic}
  \begin{itemize}
  \item{First-order logic: variables + logical connectives 
    + inference rules + axioms \textbf{+ quantifiers ($\forall$, $\exists$) 
      + (often) equality ($=$) \\ 
      + predicates + functions} }
    \medskip
  \item{Satisfiability modulo theories (SMT): is there an assignment of variables to values such that the formula evaluates to true?}
    % generalization of SAT
  \end{itemize}
\end{frame}

\begin{frame}
  \frametitle{Example: (first-order) theory of integers}
  Insert simple integer programming problem, one that could be solved with Gaussian elimination
\end{frame}


 \begin{frame}
\frametitle{Can we automate program analysis tasks?}
\begin{itemize}
  \item{Yes! One way: build analysis systems using SMT solvers.}
    \medskip
  \item{Example theories: integers, bit-vectors, arrays}
      \end{itemize}
  \end{frame}

  \begin{frame}
\frametitle{A more realistic example}
  Show angr to find buffer overflow?
  \end{frame}

\begin{frame}
\frametitle{Where can mathematicians help?}
\begin{itemize}
\item{Modeling}
  \medskip
\item{Search strategies}
  \medskip
\item{Incorporating information from other analyses}
  \end{itemize}
\end{frame}


\end{document}
